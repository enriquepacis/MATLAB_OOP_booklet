\appendix

% ============================================================
% ============================================================
% ============================================================
\section{Control Statements}
% ============================================================
% ============================================================
% ============================================================


% ============================================================
% ============================================================
\subsection{\texttt{if}-\texttt{elseif}-\texttt{else}-\texttt{end}}
% ============================================================
% ============================================================
Perhaps the simplest control statement is the achieved using \texttt{if}-\texttt{elseif}-\texttt{else}-\texttt{end}. The simplest version of this is an \texttt{if}-\texttt{end} statement, with the following typical syntax in a script or function:
% vvv------------------------------------------------------------vvv
\begin{lstlisting}[style=Matlab-editor]
if ctrl_expr
	statements
end
\end{lstlisting}
% ^^^------------------------------------------------------------^^^
Here, the control begins with \verb!if ctrl_expr! and ends with the key word \texttt{end}. The statements are executed if the real part of the control expression \verb!ctrl_expr! has all non-zero elements.

Most typically, \verb!ctrl_expr! is an expression which evaluates to a \texttt{logical} or a \texttt{double} value. An example of this is
% vvv------------------------------------------------------------vvv
\begin{lstlisting}[style=Matlab-editor]
% basic_if_control.m

a = true
if a
    disp('''a'' is true.')
end
\end{lstlisting}
% ^^^------------------------------------------------------------^^^
The output of \verb!basic_if_control.m! is
% vvv------------------------------------------------------------vvv
\begin{lstlisting}[style=Matlab-editor]
>> basic_if_control

a =

  logical

   1

'a' is true.
\end{lstlisting}
% ^^^------------------------------------------------------------^^^
Notice that \texttt{a} is a logical value, and it results in the execution of the block within the \texttt{if}-\texttt{end} construct.

Optionally, the \texttt{else} key word adds a section which is executed if the preceding sections of the \texttt{if} control is not executed. The \texttt{if}-\texttt{else}-\texttt{end} syntax is demonstrated as follows.

As an example of this, consider the function \verb!sign_fun()!:
% vvv------------------------------------------------------------vvv
\begin{lstlisting}[style=Matlab-editor]
function sign_fun(x)
% sign_fun(x) prints a message about the negativity of x.
%
% By E.P. Blair
% Baylor University

if x < 0
    disp('x is negative.')
else
    disp('x is non-negative.')
end
\end{lstlisting}
% ^^^------------------------------------------------------------^^^
This function was invoked on the command line with several test inputs:
% vvv------------------------------------------------------------vvv
\begin{lstlisting}[style=Matlab-editor]
>> sign_fun(-5)
x is negative.
>> sign_fun(-1.25)
x is negative.
>> sign_fun(0)
x is non-negative.
>> sign_fun(1)
x is non-negative.
>> 
\end{lstlisting}
% ^^^------------------------------------------------------------^^^

The \texttt{if}-\texttt{else}-\texttt{end} syntax allows for two different outcomes. Optional \texttt{elseif} blocks may be added to support additional possible outcomes. Consider, for example, this improved version of \verb!sign_fun()!:
% vvv------------------------------------------------------------vvv
\begin{lstlisting}[style=Matlab-editor]
function sign_fun(x)
% sign_fun(x) prints a message about the sign of x.
%
% By E.P. Blair
% Baylor University

if x < 0
    disp('x is negative.')
elseif x > 0
    disp('x is positive.')
else
    disp('x is zero.')
end
\end{lstlisting}
% ^^^------------------------------------------------------------^^^
The command line inputs and outputs for several tests are shown below.
% vvv------------------------------------------------------------vvv
\begin{lstlisting}[style=Matlab-editor]
>> sign_fun(-2)
x is negative.
>> sign_fun(3)
x is positive.
>> sign_fun(0)
x is zero.
\end{lstlisting}
% ^^^------------------------------------------------------------^^^

Multiple \texttt{elseif} blocks may be added, as in the following function, which assigns a letter grade given a percentage score:
% vvv------------------------------------------------------------vvv
\begin{lstlisting}[style=Matlab-editor]
function lg = letter_grade(percentScore)
% letter_grade maps a percentage score to a letter grade.

if percentScore < 60
    lg = 'F';
elseif percentScore < 70
    lg = 'D';
elseif percentScore < 80
    lg = 'C';
elseif percentScore < 90
    lg = 'B';
else
    lg = 'A';
end
\end{lstlisting}
% ^^^------------------------------------------------------------^^^

% ============================================================
% ============================================================
% ============================================================
\subsection{\texttt{for} Loops}
% ============================================================
% ============================================================
% ============================================================

% ============================================================
% ============================================================
% ============================================================
\subsection{\texttt{switch}-\texttt{case} Controls} \label{appendix:switch_case}
% ============================================================
% ============================================================
% ============================================================

A \texttt{switch}-\texttt{case} is useful when a finite, discrete set of cases may occur. The syntax for a \texttt{switch}-\texttt{case} control is as follows:
% vvv------------------------------------------------------------vvv
\begin{lstlisting}[style=Matlab-editor]
switch expr
	case expr_1
		statement_group_1
	case expr_2
		statement_group_2
	...
	otherwise
		statement_group_otherwise
end
\end{lstlisting}
% ^^^------------------------------------------------------------^^^
Here, the controlling expression evaluates to either a \texttt{char} string or an integer. If \texttt{expr} matches \verb!expr_1!, then only \verb!statement_group_1! executes; if \texttt{expr} matches \verb!expr_2!, then only \verb!statement_group_2! executes. The \texttt{otherwise} key word defines another block of statements that executes if \texttt{expr} does not match any of the expressions following a \texttt{case} key word. \texttt{Asset/calculateValue()} avoids the calculation if the \texttt{Asset} object's \texttt{transactionList} property is empty by using an \verb!if ~isempty(obj.transactionList)! block. Thus, the block executes if \texttt{obj.transactionList} is not empty. Inside this block, a \texttt{for} loop iterates through each transaction and calculates/records the number of units transacted. \texttt{Asset/calculateValue()} uses a the price-per-unit data to determine the transaction cost, the total value, and the cost basis for the asset holdings.

 
